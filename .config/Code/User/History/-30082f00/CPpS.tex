\documentclass{article}

\usepackage[english]{babel}
\usepackage{amsthm}        
\newtheorem{theorem}{Theorem}

\usepackage{mathtools}
\DeclareMathOperator\supp{supp}
\usepackage{amsmath}
\usepackage{amsfonts}
\usepackage{bbm}
\newtheorem{corollary}{Corollary}[theorem]

\author{Edward Jee}
\title{Multiple Hypothesis Testing}
\begin{document}
\maketitle


\section*{}
Suppose, we want to test multiple hypotheses simultaneously. We could be 
interested in multiple properties of a distribution or want to control the 
type 1 error rate across multiple inferences\footnote{See the appendix at the 
end of this note for more details.}

    
Let $\theta, \theta_0 \in \mathbbm{R}^d, $ for some integer $d \in \mathbbm{N},$ and consider\begin{align}
        H_0: \theta = \theta_0\quad \text{versus} \quad H_1: \theta = \theta_0.
\end{align}
    
    Because $\theta$ is a vector, we cannot proceed with the test-statistics considered thus far. 
    \subsubsection*{Preparation}
    First, let's add some tools to adapt the continuous mapping theorem to 
    statements about convergence in distribution:
    
    \begin{theorem}\label{cmt-d}
        Let $X_1, ..., X_n$ and $X$ be random vectors in $\mathbbm{R}^m$. Suppose 
        that $g \ : \mathbbm{R}^m \rightarrow \mathbbm{R}^d$ is continuous at 
        each point in the set $C$ such that $P(X \in C) = 1$. Then:
        \begin{align*}
            X_n \xrightarrow{d} X \implies g(X_n) \xrightarrow{d} g(X)
        \end{align*}
    \end{theorem}


    We've come across this before! Specifically, Slutsky is the case where 
    $X_n \xrightarrow{d} X, Y_n \xrightarrow{d} c$ for some constant $c$ and
     since addition and 
    multiplication are continuous operations we have $X_n' Y_n \xrightarrow{d} X'c$, 
    $X_n + Y_n \xrightarrow{d} X + c$. 


    Next, we recall Corollary 3 from Lecture 2A, repeated below as Corollary \ref{chisq}. 
    \begin{corollary}\label{chisq}
    Let $X \sim N(\mu, \Sigma)$ where $\supp X = \mathbbm{R}^m$. Then \begin{align}
        (X - \mu)^\top \Sigma^{-1}(X - \mu) \sim \chi^2(m).
    \end{align}  
    \end{corollary}
    \subsubsection*{Constructing a Test Statistic}
    Let $\hat{\theta}_n$ be an estimator for $\theta$ such that \begin{align}\label{sl_5:eq_normallimmutlivar}
        \sqrt{n}\hat{\Sigma}_n^{-\frac{1}{2}}\left(\hat{\theta}_n - \theta\right) \overset{d}{\to} N(0, I_d),
    \end{align}
    where $\hat{\Sigma}_n^{-\frac{1}{2}}$ is a consistent estimate of $\Sigma^{-\frac{1}{2}}: \Sigma^{-\frac{1}{2}} \Sigma^{-\frac{1}{2}} = \Sigma$, Then,
    \begin{align}\label{sl_5:eq_tnmultivar}
        T_n = n\left(\hat{\theta}_n - \theta_0\right)^\top\hat{\Sigma}_n^{-1}\left(\hat{\theta}_n - \theta_0\right).
    \end{align}
    $T_n$ captures the squared deviations of $\hat{\theta}_n$ from $\theta_0$ scaled by $\hat{\Sigma}_n^{-1}.$

    We can leverage large sample theory to show that $T_n$ is asymptotically $\chi^2$ distributed with $d$ degrees of freedom.
    
    \begin{corollary}\label{sl_5:cor_chi2}
     Let  $\hat{\theta}_n$ be an estimator for $\theta \in \mathbbm{R}^d$ such that \eqref{sl_5:eq_normallimmutlivar} holds. Then for $T_n$ defined by  \eqref{sl_5:eq_tnmultivar}, it hold that \begin{align}
            T_n \overset{d}{\to} \chi^2(d).
        \end{align}
    \end{corollary}


    This is great news - we now have the tools to calculate a test statistic and 
    we know its distribution under the null hypothesis.

    

    Corollary \ref{sl_5:cor_chi2} can then be used to show that $T_n$ is a useful test statistic:
    
    \begin{theorem}
     Let  $\hat{\theta}_n$ be an estimator for $\theta \in \mathbbm{R}^d$ such that \eqref{sl_5:eq_normallimmutlivar} holds. Then for $T_n$ defined by  \eqref{sl_5:eq_tnmultivar}, it hold that \begin{align}
            P(T_n > c_{d, 1-\alpha}\vert \: H_0 \text{ is true}) \to \alpha,
        \end{align}
        where $c_{d, 1-\alpha}$ is the $1-\alpha$ quantile of a $\chi^2$ distribution with $d$ degrees of freedom.
    \end{theorem}

    \subsubsection*{Testing our Hypotheses}

    \textbf{Step 1:}
    $Z_n = \sqrt{n}\Sigma^{-\frac{1}{2}} ( \hat{\theta}_n - \theta_0)$
    \\
    \textbf{Step 2:}
    Define $g(a) = a\top a$
    \\
    \textbf{Step 3:}
    $Z_n \xrightarrow{d} N(0, I_d)$
    \\
    \textbf{Step 4:}
    By our new CMT, Theorem \ref{cmt-d} for statements about convergence in distribution:
    \begin{align*}
        g(Z_n) &\xrightarrow{d} g(Z) \\ 
        &= Z^\top Z \\
        &= \begin{pmatrix}
            z_1 & z_2 & ... & z_d
        \end{pmatrix}
         \begin{pmatrix}
            z_1 \\
            z_2 \\
            \vdots \\
            z_d
        \end{pmatrix} \\
        &= \sum^d_{i=1} z_i^2 \sim \chi^2(d)
    \end{align*}



    \subsection*{Why not test each dimension one-by-one?}

    Let's think about the size of the test. In Theorem 2 we established the test 
    size would be equal to $\alpha$ in the limit. Now consider an alternative procedure, 
    for each element of $\hat{\theta}$ we test the univariate hypothesis:
    
    \begin{align*}
        T_n^d = \frac{\sqrt{n}|\hat{\theta}^d - \theta^d|}{\sqrt{\Sigma_{d,d}}}
    \end{align*}

    $P(T_n > c_{d, 1-\alpha}\vert \: H_0 \text{ is true}) \to \alpha$
    % %%%%%%%%%%%%%%%%%%%%%%
    % We can leverage large sample theory to show that $T_n$ is asymptotically $\chi^2$ distributed with $d$ degrees of freedom.
    
    % \begin{corollary}\label{sl_5:cor_chi2}
    %  Let  $\hat{\theta}_n$ be an estimator for $\theta \in \mathbbm{R}^d$ such that \eqref{sl_5:eq_normallimmutlivar} holds. Then for $T_n$ defined by  \eqref{sl_5:eq_tnmultivar}, it hold that \begin{align}
    %         T_n \overset{d}{\to} \chi^2(d).
    %     \end{align}
    % \end{corollary}
    
    % \begin{proof}
    % \vspace{7em}
    % \end{proof}
    
    % \vfill
    
    % %%%%%%%%%%%%%%%%%%%%%%
    
    
    % \begin{proof}
    % \vspace{7em}
    % \end{proof}
    
    % \vfill
    
\end{document}