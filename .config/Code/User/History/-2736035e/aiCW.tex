\documentclass[twoside,11pt]{article}

% Any additional packages needed should be included after jmlr2e.
% Note that jmlr2e.sty includes epsfig, amssymb, natbib and graphicx,
% and defines many common macros, such as 'proof' and 'example'.
%
% It also sets the bibliographystyle to plainnat; for more information on
% natbib citation styles, see the natbib documentation, a copy of which
% is archived at http://www.jmlr.org/format/natbib.pdf

\usepackage{jmlr2e}
\usepackage{amsmath}
\usepackage{bm}
\usepackage{algorithm}
\usepackage{algpseudocode}
\usepackage{booktabs}
% Definitions of handy macros can go here

\newcommand{\dataset}{{\cal D}}
\newcommand{\fracpartial}[2]{\frac{\partial #1}{\partial  #2}}

% Heading arguments are {volume}{year}{pages}{submitted}{published}{author-full-names}

\jmlrheading{1}{}{}{}{}{}

% Short headings should be running head and authors last names

%\ShortHeadings{Learning with Mixtures of Trees}{Meil\u{a} and Jordan}
\firstpageno{1}

\begin{document}

\title{
  Treatment and Welfare Learning for Policymakers
}

\author{\name Edward Jee \email edjee@uchicago.edu \\
       \addr Kenneth C. Griffin Department of Economics\\
       University of Chicago\\
       Chicago, IL 60637, USA}


\maketitle

\begin{abstract}%   <- trailing '%' for backward compatibility of .sty file
I describe a modified Thompson sampling algorithm that jointly learns participant preferences and treatment 
effects to estimate the welfare optimal treatment arm in an adaptive field 
experiment. In contrast to \cite{lin2022preference}, I design an incentive 
compatible mechanism to reveal \emph{participant} preferences and not those of 
the decision maker, even in the presence of strategic behaviour by participants 
or scepticism of information 
provision as is common in development economics. Finally, by randomising 
participants into their preferred choice or alternative choices I can directly 
decompose the average treatment effect (ATE) vectors into an average treatment 
effect on the treated (ATT) and average treatment effect on the untreated (ATU); 
aggregate these treatment effects into scalar welfare measures; and demonstrate 
the importance of selection when policymakers design interventions.
\end{abstract}

\begin{keywords}
 Multi-Objective Decision Making, Participant Preferences, Incentive Compatibility, 
 Development Economics, Programme Evaluation, Causal Inference 
\end{keywords}

\section{Introduction}

Field experiments in development economics often observe multiple, common outcomes 
across treatment arms and must identify an optimal arm for a policymaker. 
 When using adaptive trials economists typically maximise a single outcome or 
 some standardised index. Taking such a decision seriously 
implies the economist's imposed utility function reflects participants' preferences 
and these preferences only depend on one feature of the outcome vector. Alternatively, 
using a standardised index implies 
participants value the marginal value of a variance weighted good equally across 
outcomes\footnote{For a static example see \cite{ashraf2010a,blattman2017a,bandiera2017a}}. Instead, I propose an algorithm that jointly learns participant 
preferences and treatment arm effectiveness whilst also identifying exactly the 
policy parameters of interest - an average treatment effect (ATE) vector, which 
can further be decomposed into an average treatment effect on the treated (ATT) and 
an average treatment effect on the untreated (ATU) and finally, the necessary marginal 
rates of substitution to 
aggregate vector valued treatment effects into scalar units of  welfare. 


My results are most similar to \citep{dewancker2016a,lin2022preference}, the latter seeks 
to learn a decision maker's utility function using pairwise comparisons when a 
time-consuming or expensive experiment must be run to learn 
vector valued outcomes. My setting differs in two key ways: First, it is the 
participants' preferences we wish to learn and not the experimenter's. Second, 
instead of simply identifying the ATE, my proposed algorithm identifies both 
the ATT and the ATU. These parameters are often more relevant to the policymaker's 
decision to implement the programme considered.


Moving from decision maker's preferences to participants' introduces several 
complications. A rich literature in economics, pyschology, and experimental fields 
more generally demonstrate 
the importance of designing mechanisms such that participants are incentivised 
to reveal their true preferences \citep{Savage1971,ar-delavande}. When the decision maker and experimenter are 
one and the same, 
incentives are clearly aligned and stated preferences can be treated as the ground 
truth. However, when the experimenter must learn the decision maker's 
preferences he/she must account for the possibility of strategic behaviour and 
"cheap talk". For instance, participants may misreport their preferences in a bid
to increase their expected payoff in response to other participants' choices, or 
social desirability bias may lead to participants stating preferences to match 
the experimenter's expectations or social norms \citep{brownback2018a}. Therefore, my work 
differs from \cite{lin2022preference} and methods outlined by \citep{furnkranz2010a} by proposing 
an incentive compatible 
preference elicitation step for participants by allowing individuals
to rank treatment arms and assign subjects to arms with a probability
concordant with their stated preferences. In some sense, my approach "disciplines"
participants by ensuring stated preferences have consequences.


Furthermore, whilst \cite{lin2022preference} treat output vectors from the surrogate 
model as given and elicit preferences across these vectors, development economists 
face an environment where information signals are treated with 
scepticism and both treatment \textit{and} preference experimentation/elicitation
 in the field is costly. Therefore, I propose a second algorithm with a ``structural" 
 model of belief formation that maps posterior signals, which can be imparted by 
 the experimenter cheaply in the field, to participant posterior beliefs.

% Furthermore, whilst \cite{lin2022preference} treat output vectors from the surrogate 
% model as given and elicit preferences across these vectors, development economists 
% face a more challenging environment where information signals are treated with 
% scepticism \citep{conlon-schilbach,mobius-sc}. Furthermore, psychologists and behavioural economists have shown a 
% variety of behavioural biases exist that must be considered when mapping information 
% signals to posterior beliefs such as inattention \citep{sparse-ration,GABAIX2019261}, learning through noticing \citep{learning-noticing}, 
% and conservatism bias \citep{BENJAMIN201969}. Therefore, I show how to elicit posterior beliefs in more challenging 
% environment likely to be faced in the field by experimenters and microfound 
% choices using the random utility model (RUM) of \cite{mcfadden}.

Finally, this paper seeks to further embed the principles of the Belmont report, 
published in 1979 after numerous human subject research violations, in randomised 
control trials run by development economists and other researchers. By enshrining 
participant preferences at the center of randomised trials, my proposed algorithm 
speaks directly to respect for persons; beneficence; and justice as outlined by 
the report. Respect for participants and their preferences, who in development 
economics often reside in low-income countries, is particularly important in 
a field dominated by rich, US-based academics \citep{stansbury} running trials on those with 
little agency, income, or human capital.

\section{Setup and Method}

In this section, I will briefly describe two algorithms. The first assumes a benign 
enviroment without budget constraints and elicits posteriors directly. The second focuses on a more 
realistic field environment often faced by development economists in low-income
countries where information can be costly to obtain.

The experimenter faces a vector of outcomes $Y_{i} = \begin{bmatrix}
    y^1_{i} & y^2_{i} & ... & y^j_{i}
\end{bmatrix}'$, $Y_i \in \mathbb{R}^j$ for individual $i$ and must determine 
the optimal treatment 
arm $k$ with associated reward vector $\bm{\mu}_k = \begin{bmatrix}
    \mu^1_k & \mu^2_k & ... & \mu^j_k
\end{bmatrix}', \bm{\mu}_k \in \mathbb{R}^j$, individuals arrive in waves of size
$N_t$. Individuals have knowledge about each treatment arm's 
expected effectiveness on their own outcomes. Participant's
 utility functions are parametrised using \cite{mcfadden}'s discrete 
choice random utility model\footnote{Any non/semi/fully-parametric choice model could be used here 
but economists are particularly partial to the multinomial/rank-ordered logit.}: $U_k = V_k + \varepsilon_k$ 
where $U_k$  corresponds to the utility an individual receives from \emph{choosing} treatment arm $k$ 
with $V_k = \gamma_1 \mu_k^1 + \gamma_2 \mu_k^2 + ... + \gamma_j \mu_k^j = \bm{\mu}_k \bm{\gamma}'$. 
Therefore, a participant, if offered 
the choice of treatment arms, will only choose arm $l$ if $\varepsilon_n < (V_l - 
V_n) + \varepsilon_l, \forall n \neq l$ where $\varepsilon_k \sim T1EV$. Note that 
$\gamma_j$ isn't indexed by $k$ - individuals must value a marginal gain in 
outcome $j$ the same in arm $k$ as arm $l \neq k$.


\begin{algorithm}
  \caption{Treatment and participant preference estimation }\label{alg:cap}
  \begin{algorithmic}
  \State Generate a prior $(Q_0, F_0)$ over $\left(\bm{\mu}_k, \bm{\gamma}\right)$
  \For{$t = 1,...,T$}
  % \State Sample $\bm{\nu}_{t} \sim Q_{t-1}(\cdot | p_{k}^{t-1}, Y^{t-1})$ and inform each 
  % participant of a single $\mu_k(\nu_{t})$ draw
  \State Elicit participant posterior beliefs $\bm{\mu}_k$ using a binarized scoring rule
  \State Observe participant rankings, $K_t$ and update $F_{t}( \cdot | K^{t}, \bm{\mu}_k(\bm{\nu}^{t}))$ 
  \State Sample $\bm{\omega}_{t} \sim F_{t}( \cdot | K^{t}, \bm{\mu}_k(\bm{\nu}^{t})), \bm{\nu}_t \sim Q_{t-1}(\cdot | p_{k}^{t-1}, Y^{t-1})$ 
  \State Choose $p_k = \frac{1}{N_t} 
  \sum^{N_t}_{n=1} \mathbb{I}\{\bm{\mu}_k(\nu_{t}) \bm{\gamma}(\omega_{t})' > \bm{\mu}_l(\nu_{t}) \bm{\gamma}(\omega_{t})'\}, k \neq l$ 
  \State Assign participants to treatment arm $k$ with probability $p_{k}$ using a strategy proof mechanism 
  \State Observe $Y_t$ and update the posterior $Q_{t}(\cdot | p_k^{t-1}, Y^{t})$ over $\bm{\mu}_k$. 
  \EndFor
  \end{algorithmic}
  \end{algorithm}

Each wave the experimenter elicits participant posterior beliefs about treatment 
arm effects using a \textit{binarized scoring rule} (BSR) \citep{hossain}. Under a BSR the participant 
receives a fixed reward when their prediction error is less than some independently 
generated random number. Since the reward size is fixed the individual is 
incentivised to report her true beliefs even under a range of risk preferences - only the 
probability of reward is determined by the realised score and not its size. \cite{hossain} 
show that even if a particpant's decision cannot be rationalised by expected 
utility theory, provided participant preferences satisfy a monotonicity condition 
the BSR will be incentive compatible and she will report her true beliefs.

Next, the experimenter instructs participants to rank their preferred treatment 
arms. Again, incentive compatibility is ensured by using a strategy proof mechanism
with probability of arm assignment increasing in participant rankings. One example 
would be the \textit{random serial dictatorship} mechanism whereby participants 
are randomly ordered from 1 to $N_t$, assign the first participant their first choice,
the next participant their top choice amongst the remaining choices, and so on. 
Each treatment arm accepts remaining participants until their assignment proportion, 
$p_k$\footnote{I describe $p_k$'s calculation shortly. $p_k$ doesn't need to be known 
before eliciting rankings, only when treatment is assigned.}, is reached. Random 
serial dictatorship is a commonly used school choice algorithm, partly because of 
its strategy proofness \citep{abdulkadirolu2003a}.  With rankings 
and participant posteriors in hand the experimenter estimates a rank-ordered logit, 
updating $F_t$, 
where the probability of a given ranking takes the familiar functional form:

\begin{align*}
  Pr(r_i | \bm{\gamma}) &= \prod^{K-1}_{k=1} \frac{\exp(V_{ir_{ik}})}{\sum^K_{l=k}\exp(V_{ir_{il}})}
\end{align*}


To calculate assignment probabilities the experimenter draws from the treatment
effect, $Q_{t-1}$, and discrete choice model posterior, $F_t$, to generate $\bm{\mu}_k(\nu_t), \bm{\gamma}(\omega_t)$ 
draws. Taking the linear combination of these draws, 
$\bm{\mu}_k(\nu_{t}) \bm{\gamma}(\omega_{t})'$, gives posterior arm welfare and 
$p_k$ is chosen using probability matching in proportion to the probability an 
arm's welfare is highest. Finally, the experimenter assigns participants to 
treatment arms, observes $Y_t$ and updates their treatment effect posterior, $Q_t$.

\subsection{Algorithm 2}
In many settings eliciting posteriors directly is uneconomical. Enumerators must 
be trained in the application of BSR and administer the test in the field for 
each subject \citep{glennerster}. In contrast, information treatments are cheap and easy to administer 
via text \citep{banerjee2021a}. Therefore, I propose an alternative algorithm that uses signals, 
in the form of treatment effect posterior draws, 
and a structural model of belief formation to estimate marginal rates of 
substitution across outcomes.

\begin{algorithm}
  \caption{Treatment and structural participant preference estimation }\label{alg:cap}
  \begin{algorithmic}
  \State Generate a prior $(Q_0, F_0, \Pi_0)$ over $\left(\bm{\mu}_k, \bm{\gamma}, (\bm{\mu}_0, \bm{\tau}_0^{-1})\right)$
  \For{$t = 1,...,T$}
  \If{$e_i < \alpha_t, e_i \sim U(0,1)$}
  \State Elicit participant priors, $\bm{\mu}_0, \bm{\tau}_0^{-1}$, using BRS and update $\Pi_t$
  \EndIf
  \State Sample $\bm{\nu}_{t} \sim Q_{t-1}(\cdot | p_{k}^{t-1}, Y^{t-1})$ and inform each 
  participant of a single $\mu_k(\nu_{t})$ draw
  \State Observe participant rankings, $K_t$, and update $F_{t}( \cdot | K^{t}, \bm{\mu}_k(\bm{\nu}^{t}))$ 
  given $\Pi_t$
  \State Sample $\bm{\omega}_{t} \sim F_{t}( \cdot | K^{t}, \bm{\mu}_k(\bm{\nu}^{t})), \bm{\nu}_t \sim Q_{t-1}(\cdot | p_{k}^{t-1}, Y^{t-1}), \bm{u} \sim U(0,1)$ 
  \State Choose $p_k = \frac{1}{N_t} 
  \sum^{N_t}_{n=1} \mathbb{I}\{\bm{\mu}_k(\nu_{t}) \bm{\gamma}(\omega_{t})' > \bm{\mu}_l(\nu_{t}) \bm{\gamma}(\omega_{t})'\}, k \neq l$ 
  \State Assign participants to treatment arm $k$ with probability $p_{k}$ using a strategy proof mechanism
  \State Observe $Y_t$ and update the posterior $Q_{t}(\cdot | p_k^{t-1}, Y^{t})$ over $\bm{\mu}_k$. 
  \EndFor
  \end{algorithmic}
  \end{algorithm}

In Algorithm 2 the experimenter chooses a subsample to elicit participants'
 priors over treatment arm 
effectiveness\footnote{Economists typically ask individuals to allocate beans or 
stones in intervals to generate belief distributions e.g. see \cite{ar-delavande} for more details.}
and samples $N_t$ draws, \emph{or signals,} from the joint posterior of treatment arm effects. 
Next, the experimenter individually informs participants of a private signal
and asks the individual to rank his/her preferred treatment arms. Estimating a 
rank-ordered discrete choice model of rankings on signals and normalising 
estimated coefficients by the first signal coefficient gives the marginal  
rate of substitution \emph{across signals about outcomes}. Unfortunately, this 
complicates identification somewhat as we must disentangle how much an individual 
values an additional unit of an outcome from how sceptical they are about 
outcome signals. In the interest of brevity I will focus on a conjugate Gaussian 
updating model with known variance: 


\begin{align*}
    S^i  | \mu^i  &\sim N\left(\mu^i, \tau^{i,-1}\right) \\
    \mu^i &\sim N\left(\mu^i_0, \tau_0^{i,-1}\right) \\
    \implies E[\mu'^{i} | s^{i} ]   &= \mu^i_0 + (s^i - \mu^i_0) \frac{\tau^i}{\tau^i_0 + \tau^i}, 
    \\ i &= 1, ..., j 
\end{align*}
   
That is, on receiving a signal $s$ for outcome $i$ the participant updates their
 posterior in proportion to their prior precision and precision of the signal. 
 Defining $\lambda^i = \frac{\tau^i_0}{\tau^i}$ gives:
 \begin{align*}
   \frac{ds^i}{ds^l} &= \frac{\lambda^i + 1}{\lambda^l + 1} \frac{d\mu^i}{d\mu^l}
 \end{align*}

and estimated participant prior precision and marginal rates of substitution 
across signals can be mapped into marginal rates of substitution across outcomes.


Again, the experimenter can now sample from 
the joint posterior over treatment effects and marginal rates of substitution, 
construct each arm's posterior welfare, and assign treatment using probability 
matching in proportion to estimated posterior welfare.

\subsection{Policy Relevant Parameters}

Finally, the estimated parameter vector $\bm{\mu}_k$ forms an ATE, typically 
written as $E[Y(1) - Y(0)]$ using the Fisher-Neyman-Rubin-Quandt model. The 
ATE describes a policy counterfactual if random individuals were \textit{compelled}
to take or not take treatment, $D$. In reality, policymakers typically cannot 
mandate treatment in a population and must consider whether to expand or 
withdraw a programme based on the effects for those who choose, or don't choose, 
to takeup the offered treatment \citep{heckman-1,heckman-2}.

The parameters of interest in this case corresponds to an ATT $E[Y(1) - Y(0) | D = k]$, 
and ATU, $E[Y(1) - Y(0) | D = k'], k \neq k'$. By eliciting preference rankings 
but randomly assigning individuals to treatment arms I can compare outcomes 
for those that received their favoured choice and those that don't and subsequently 
uncover the ATT and ATU.



 
\section*{Simulation Results and Discussion}


Table 1 shows results from 100 Monte Carlo draws using 15 rounds of 100 
participants per wave with four treatment arms and three outcomes to 
aggregate across. Simulation parameters are drawn from:
\begin{align*}
  \bm{\gamma} &\sim N(\bm{0}, I_3) \\
  \bm{\mu}_k &\sim N(\bm{0}, I_3), \ k = 1, ..., 4 \\
  \eta_i &\sim N(0, 1),\ \varepsilon_{ki} \sim T1EV
\end{align*}

where $\eta_i, \varepsilon_{ki}$ represent participant-level outcome and 
ranking errors respectively. Models are estimated in Stan \citep{carpenter2017a}.
\input{final-sim.tex}
Assignment type ``Estimated'' corresponds to Algorithm 1 outlined above and 
identifies the optimal arm by the end of the trial 95\% of the time. In contrast, 
static random assignment only identifes the optimal arm in 87\% of draws. ``Equal''
corresponds to Thompson sampling maximising a standardised index of the three 
outcomes whilst ``First'' only targets the first element of the outcome vector to 
maximise. Since I use a closed form solution for participant utility, using the 
multinomial logit and generated $\bm{\gamma}$ parameters, I calculate average 
welfare across participants within a simulated draw and rank each algorithm, 
denoted by ``Mean Welfare Rank''. As expected, Algorithm 1, which estimates 
participant preferences directly produces the greatest mean welfare for participants 
whilst static random assignment the lowest.

In conclusion, by carefulling incentivising research trial participants, through the 
use of binarized scoring rules and a strategy proof mechanism such as 
random serial dictatorship, I've shown how to estimate participant preferences
and aggregate treatment effects across disparate outcomes into a microfounded 
estimate of arm arm welfare. When direct posterior elicitation if infeasible or 
prohibitively expensive, drawing signals from the Thompson posterior over treatment 
effects is almost exactly the information we wish to impart to participants 
to generate observale variation in choices.
% Acknowledgements should go at the end, before appendices and references

% \acks{We would like to acknowledge support for this project
% from the National Science Foundation (NSF grant IIS-9988642)
% and the Multidisciplinary Research Program of the Department
% of Defense (MURI N00014-00-1-0637). }

% Manual newpage inserted to improve layout of sample file - not
% needed in general before appendices/bibliography.

\newpage


\vskip 0.2in
\bibliography{sample,machine}

\end{document}