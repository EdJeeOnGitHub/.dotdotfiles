\documentclass{article}


\begin{document}
   


Start with a classic simultaneous equation model:

\begin{align*}
    Y_1 &= X_1 \xi_1 + X_2 \pi_1 + U_1 \\
    Y_2 = X_1 \xi_2 + X_2 \pi_2 + U_2
\end{align*}

We're going to treat $Y_1$ as our endogenous variable and $Y_2$ as our outcome 
of interest. $X_1$ is a matrix of exogenous controls and $X_2$ are our instruments 
(people call these excluded instruments).

Instrumental variables/two stage least squares represents the following implied model:

\begin{align*}
    Y_2 &= X_1 \delta + Y_1 \gamma + V
\end{align*}


which is to say $Y_1$ is a function of exogenous controls and our endogenous $Y_1$. 
This is also where the term excluded instrument comes from - $X_2$ only influences 
$Y_2$ through $Y_1$.


Working through algebra and matching terms:

\begin{align*}
    Y_2 &= X_1 \delta + Y_1 \gamma + V \\
    Y_2 &= X_1 \delta + (X_1 \xi_1 + X_2 \pi_1 + U_1) \gamma + V \\
    \implies Y_2 &= X_1 (\delta + \gamma \xi_1) + X_2 \gamma \pi_1 + \gamma U_1 + V
\end{align*}

Therefore:

\begin{align*}
    \delta + \gamma \xi_1 &= \xi_2 \\
    \gamma \pi_1 &= \pi_2  \\
    \implies \\
    \begin{bmatrix}
        Y_1 & Y_2
    \end{bmatrix} &= \begin{bmatrix}
        X_1 & X_2
    \end{bmatrix}
    \begin{bmatrix}
        \xi_1 & \xi_2 \\
        \pi_1 & \pi_2
    \end{bmatrix} 
     &= \begin{bmatrix}
        X_1 & X_2
    \end{bmatrix}
    \begin{bmatrix}
        \xi_1 &  \delta + \gamma \xi_1 \\
        \pi_1 & \gamma \pi_1
    \end{bmatrix} 
\end{align*}




\end{document}