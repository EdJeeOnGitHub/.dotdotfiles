\documentclass{article}
\usepackage{amsmath}
\usepackage{hyperref}
\hypersetup{
    colorlinks=true,
    linkcolor=blue,
    filecolor=magenta,      
    urlcolor=cyan,
    pdftitle={Overleaf Example},
    pdfpagemode=FullScreen,
    }

\usepackage{listings}
\title{IV Note}
\author{Ed Jee}
\begin{document}
\maketitle
   


Start with a classic simultaneous equation model:

\begin{align*}
    Y_1 &= X_1 \xi_1 + X_2 \pi_1 + U_1 \\
    Y_2 &= X_1 \xi_2 + X_2 \pi_2 + U_2
\end{align*}

We're going to treat $Y_1$ as our endogenous variable and $Y_2$ as our outcome 
of interest. $X_1$ is a matrix of exogenous controls and $X_2$ are our instruments 
(people call these excluded instruments).

Instrumental variables/two stage least squares represents the following implied model:
\begin{align*}
    Y_2 &= X_1 \delta + Y_1 \gamma + V
\end{align*}


which is to say $Y_1$ is a function of exogenous controls and our endogenous $Y_1$. 
This is also where the term excluded instrument comes from - $X_2$ only influences 
$Y_2$ through $Y_1$.


Working through algebra and matching terms:
\begin{align*}
    Y_2 &= X_1 \delta + Y_1 \gamma + V \\
    Y_2 &= X_1 \delta + (X_1 \xi_1 + X_2 \pi_1 + U_1) \gamma + V \\
    \implies Y_2 &= X_1 (\delta + \gamma \xi_1) + X_2 \gamma \pi_1 + \gamma U_1 + V
\end{align*}

Therefore:

\begin{align*}
    \delta + \gamma \xi_1 &= \xi_2 \\
    \gamma \pi_1 &= \pi_2  \\
    \implies \\
    \begin{bmatrix}
        Y_1 & Y_2
    \end{bmatrix} &= \begin{bmatrix}
        X_1 & X_2
    \end{bmatrix}
    \begin{bmatrix}
        \xi_1 & \xi_2 \\
        \pi_1 & \pi_2
    \end{bmatrix}  \\
     &= \begin{bmatrix}
        X_1 & X_2
    \end{bmatrix}
    \begin{bmatrix}
        \xi_1 &  \delta + \gamma \xi_1 \\
        \pi_1 & \gamma \pi_1
    \end{bmatrix} 
\end{align*}

Incidentally, this is why people get uncomfortable with IV with non-linear first/second 
stages. The substitution we did above only works in a linear setting because IV 
is effectively a linear projection and so everything goes through nicely.


In Stan code this looks like \href{https://gist.github.com/EdJeeOnGitHub/f728ac156da2d8dafa4dda8cd1a6b336}{this}.

\textit{This is a really bad example as I stupidly don't have a simpler case that 
can be run quickly. The only 
thing worth focusing on is the } \textbf{create\_beta} \textit{function which creates the  
coefficient matrix we define above and sticks this in a multivariate normal 
likelihood. The rest is all sufficient statistic and horseshoe mess.}

\subsection*{Parameters in Stan}
This is why, even though some people write IV likeso:

\begin{align*}
    Y &= \hat{x} \beta + \varepsilon \\
    x &= z\pi + u
\end{align*}


That is, $\hat{x}$ is the predicted value from the first stage (the second equation), 
we don't have to treat $\hat{x}$ as a parameter in Stan. This is 
because to economists we're projecting $x$ onto $z$ which is assumed to be 
uncorrelated with $\varepsilon$ and the output of this linear projection isn't some 
paramter.
\end{document}