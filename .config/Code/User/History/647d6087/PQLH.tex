
\documentclass{article}

\setlength\textwidth{135.5mm}
\setlength\textheight{200mm}
\setlength\topmargin{-10pt}
\setlength\oddsidemargin{14mm}
\setlength\evensidemargin{\oddsidemargin}
\setlength\headheight{24pt}
\setlength\headsep   {14pt}
\setlength\topskip   {11.74pt}
\setlength\maxdepth{.5\topskip}
\setlength\footskip{15pt}



\title{Peer Review}
\author{Edward Jee}
\begin{document}
\maketitle



\section*{What I Liked}


The proposal clearly motivates the issue of agricultural technology adoption and 
how behavioural beliefs and social interactions could possibly lead to 
under-adoption. The focus on African countries, given technology adoption rates, 
is also clearly explained and even the choice of Angola in particular as a 
setting to study this problem.


I also like how the identification strategy attempts to isolate social norms 
from social learning. By focusing on harvests where farmers haven't observed 
crop yields but have potentially observed a different planting technique the 
proposal can tease out the effect of social norms. 

\section*{Points to address}


I have some points I think it'd be useful to think about, described below.


\subsubsection*{Choice of Angola}
As raised in the presentation, it would be good to investigate why Angola in 
particular has such low crop yields. It seems plausible there could be other, 
more immediate issues driving low yields and perhaps social norms could be 
better studied in another setting which we understand better?



\subsubsection*{Choice of training levels}

Currently, the control group has 10\% of farmers trained and the treatment group 
has 80\%. My understanding is that we want to shift social norms so create a 
treatment group with majority of farmers trained and a control group with a 
minority. However, it's not clear to me that social norms will only shift if we 
have >50\% of farmers trained - in some sense it could be easier to detect 
effects in the control group since we observe 90\% of the sample that could 
swap technology but only 20\% of the sample in the treatment group can be 
influenced.

Therefore, you probably want to think about a model/write down the dynamics of
social norm formation and figure out just how much fraction of farmers should 
be trained. If this is an open empirical question we don't have an answer to, 
why not create three treatment groups and vary training percentages from 25\% to 
75\% for instance. Power calculations would be useful here to figure just how 
to spread out the sample.



Why these levels of training in treat and control? 
- In fact, why any level of training in control?
- Doesn't 80\% treatment make it harder to pickup effects as we can only 
observe learning in 20\% of the remaining sample as opposed to 90\% in control?



- What if farmers get trained but don't use the training. I.e. social norms 
could be so strong they don't want to swap but the estimation strategy would then 
show a null result.

- Why do we care about social norms in this setting since farmers will always be 
able to observe crop yields and socially learn instead of socially norm. 





- Why 2 week training into 16 week training? Especially since this 16 week 
training will be the "dirty" comparison after crop yields have been observed in 
the 2 week training group.
    
\end{document}