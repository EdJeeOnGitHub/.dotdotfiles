
\documentclass{article}

\setlength\textwidth{135.5mm}
\setlength\textheight{200mm}
\setlength\topmargin{-10pt}
\setlength\oddsidemargin{14mm}
\setlength\evensidemargin{\oddsidemargin}
\setlength\headheight{24pt}
\setlength\headsep   {14pt}
\setlength\topskip   {11.74pt}
\setlength\maxdepth{.5\topskip}
\setlength\footskip{15pt}



\title{Peer Review}
\author{Edward Jee}
\begin{document}
\maketitle



\section*{What I Liked}


The proposal clearly motivates the issue of agricultural technology adoption and 
how behavioural beliefs and social interactions could possibly lead to 
under-adoption. The focus on African countries, given technology adoption rates, 
is also clearly explained and even the choice of Angola in particular as a 
setting to study this problem.


I also like how the identification strategy attempts to isolate social norms 
from social learning. By focusing on harvests where farmers haven't observed 
crop yields but have potentially observed a different planting technique the 
proposal can tease out the effect of social norms.

\section*{Points to address}


Is Angola special in particular
Why these levels of training in treat and control? 
- In fact, why any level of training in control?
- Doesn't 80\% treatment make it harder to pickup effects as we can only 
observe learning in 20\% of the remaining sample as opposed to 90\% in control?



- What if farmers get trained but don't use the training. I.e. social norms 
could be so strong they don't want to swap but the estimation strategy would then 
show a null result.

- Why do we care about social norms in this setting since farmers will always be 
able to observe crop yields and socially learn instead of socially norm. 





    
\end{document}