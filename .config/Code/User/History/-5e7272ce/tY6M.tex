\documentclass{article}
\author{Ed Jee}
\title{Summary: Prosociality and the Quest for Talent}
\usepackage[margin=0.5in]{geometry}
\setlength\textwidth{135.5mm}
\setlength\textheight{200mm}
\setlength\topmargin{-10pt}
\setlength\oddsidemargin{14mm}
\setlength\evensidemargin{\oddsidemargin}
\setlength\headheight{24pt}
\setlength\headsep   {14pt}
\setlength\topskip   {11.74pt}
\setlength\maxdepth{.5\topskip}
\setlength\footskip{15pt}

\begin{document}
\maketitle 

% (i) Why is the paper important (or why not)? (ii) An
% overview of the core contributions of the paper. (iii) What you liked – or did not like – about
% the paper

\section*{Contributions}


Ashraf et al. investigate the optimal strategy for the recruitment of public 
healthcare workers whilst trying to balance incentives for talent with prosociality 
in Zambia.
There are concerns that raising pay for community healthcare jobs will attract job applicants 
with low prosociality but greater career concerns/money motivation etc. - this 
is undesirable for the government since prosocial workers tend to be more 
productive than their career-minded counterparts but higher wages attract 
more talented workers. The paper tests this 
hypothesis directly by randomly allocating districts to a treatment and control 
group. The treatment group received job information posters that emphasised the 
position entailed joining the civil service; described work colleagues as 
health experts; and the possibility of career advancement. However, the control 
group recieved job posters emphasising the ability to help the community and 
other prosocial aspects of the job. In line with predictions from theory, 
the average applicant was more talented and less prosocial in the treatment 
group.


In order to separate treatment induced incentives to perform their jobs better, 
the programme used a year of training with the same regime and information for all applicants.
A survey at the end of the training programme confirmed beliefs about the job 
had converged.


\end{document}