%----------------------------------------------------------------------------------------
%	PACKAGES AND THEMES
%----------------------------------------------------------------------------------------
\documentclass[aspectratio=169,xcolor=dvipsnames]{beamer}
\usetheme{SimplePlus}

\usepackage{hyperref}
\usepackage{bm}
\usepackage{graphicx} % Allows including images
\usepackage{booktabs} % Allows the use of \toprule, \midrule and \bottomrule in tables
\setbeamertemplate{footline}{%
  \raisebox{5pt}{\makebox[\paperwidth]{\hfill\makebox[10pt]{\scriptsize\insertframenumber}}}}

%----------------------------------------------------------------------------------------
%	TITLE PAGE
%----------------------------------------------------------------------------------------

\title[short title]{Adaptive Trials and Participant Preferences} % The short title appears at the bottom of every slide, the full title is only on the title page
\subtitle{}

\author[Edward Jee] {Edward Jee}

\institute[NTU] % Your institution as it will appear on the bottom of every slide, may be shorthand to save space
{
Development Tea
}
\date{\today} % Date, can be changed to a custom date


%----------------------------------------------------------------------------------------
%	PRESENTATION SLIDES
%----------------------------------------------------------------------------------------

\begin{document}

\begin{frame}
    % Print the title page as the first slide
    \titlepage
\end{frame}

\begin{frame}{Today's Plan}
    % Throughout your presentation, if you choose to use \section{} and \subsection{} commands, these will automatically be printed on this slide as an overview of your presentation
    \tableofcontents
\end{frame}

%------------------------------------------------
\section{Adaptive Trials}
%------------------------------------------------

\begin{frame}{What is an Adaptive Trial?}

    \begin{itemize}
        \item An experimental trial where we change treatment assignments \textit{depending on the realisation of observed treatment outcomes.}
        \item Each period we receive a new wave of participants and update treatment arm allocations 
        using results from previous participants.
        \item The goal is to maximise welfare of treatment participants, by learning 
        about the optimal arm as quickly as possible. 
        \item In computer science this is termed the \textit{multi-armed bandit} problem.
    \end{itemize}
\end{frame}

%------------------------------------------------

\begin{frame}{Example Bandit}
    \textit{And why are they called bandits?}
    \begin{figure}[htbp]
        \centering
        \includegraphics[scale=0.1]{plots/bandit-image-example.png} 
        \caption{Some real life bandits}
        \label{<label>}
    \end{figure}
\end{frame}
%------------------------------------------------

\begin{frame}{Thompson Sampling}

    A popular bandit algorithm is known as Thompson 
    sampling:\footnote{Also known as probability matching.}
\begin{enumerate}
    \item At $t=0$ generate a prior, $Q_0$, over $\mu_k$, proceed to $t = 1$.
    \item Sample $\nu_t \sim Q_{t-1}(\cdot | A_1, y_1, ..., A_{t-1}, y_{t-1})$
    \item Choose $A_t = \arg\max_{i \in [k]} \mu_i(\nu_t)$ and observe $y_t$.
    \item Update the posterior $Q_{t-1}(\cdot | A_1, y_1, ..., A_{t-1}, y_{t-1})$ given $y_t$.
    \item Iterate $t = t+1$ and return to Step 2 until $t > T$.
\end{enumerate}

\end{frame}


\begin{frame}

    
    In words:

    \begin{itemize}
        \item Generate a prior over each arm's rewards.
        \item Take a random draw from this prior distribution and choose the best action given your draw.
        \item Observe the reward associated with your action.
        \item Update your posterior over arm rewards.
        \item Today's posterior is tomorrow's prior $\rightarrow$ return to step two.
    \end{itemize}

\end{frame}


\begin{frame}{Exploration-Exploitation Tradeoff}
    Why do we like Thompson sampling?

    \begin{itemize}
        \item It neatly balances the exploration-exploitation tradeoff.
        \item We don't want to spend all our time trying each arm in turn if one arm 
        is clearly suboptimal.
        \item Similarly, we don't want to commit too early to an arm and exclusively 
        choose it when it's not the best arm. 
    \end{itemize}

    TS updates posterior beliefs each round and takes a \textit{random draw} from 
    the posterior. This means we will choose to explore another arm over the 
    "currently optimal" arm in proportion to the exact probability we believe it 
    could be optimal.
\end{frame}



\begin{frame}{Beta-Bernoulli TS}

    Suppose we face a Bernoulli bandit so $\mu \in [0,1]^k$ but 
    each arm has a different probability of success. 


    Learner has an independent beta prior with parameters $\alpha_i, \beta_i$:
    \begin{align*}
        p(\mu_i) &= \frac{\Gamma(\alpha_i + \beta_i)}{\Gamma(\alpha_i) \Gamma(\beta_i)} 
        \mu_i^{\alpha_i - 1} (1 - \mu_i)^{\beta_i - 1}
    \end{align*}

    Each round we update:
    \begin{align*}
        (\alpha_{i, t+1}, \beta_{i, t+1}) =\begin{cases}
            (\alpha_{i, t}, \beta_{i, t}) &  A_t \neq i \\
            (\alpha_{i, t}, \beta_{i, t}) + (\mu_t, 1 - \mu_t) & A_t = i
        \end{cases} 
    \end{align*}
   Posterior mean:
    $$
   \hat{\mu}_i = \frac{\alpha_i}{\alpha_i + \beta_i}
   $$
    $\alpha, \beta$ are "pseudo-counts" of successes and failures.

    Due to conjugacy, the posterior will also be beta!

\end{frame}


\begin{frame}
    \frametitle{TS Example  - \href{https://towardsdatascience.com/thompson-sampling-fc28817eacb8}{Source}}

   \begin{figure}[htbp]
       \centering
       \includegraphics[scale=0.2]{plots/ts-post.png}
   \end{figure} 

\end{frame}

\begin{frame}{BB TS II}


    Compare with the "greedy" alternative which doesn't sample from the posterior, 
    merely updates $\hat{\mu}_i = \frac{\alpha_i}{\alpha_i + \beta_i}$:

    \begin{figure}[htbp]
        \centering
       \includegraphics[scale=0.25]{plots/ts-example.png} 
        \caption{Greedy vs Full Posterior Algorithm}
    \end{figure} 

\end{frame}


\section{Adaptive Trials in Economics}
\begin{frame}{Bandits in Economics I}
    \begin{itemize}
        \item Computer Scientists focus on a single learner choosing one action.
        \item We need to assign treatment proportions for multiple participants rather 
        than a single action.
        \item Therefore assign treatments in proportion to which they're optimal:
        $p_k = \frac{1}{H} \sum^H_{h=1} \mathbb{I}\{\mu_k(\nu_{h})  > \mu_l(\nu_{h}) \}, k \neq l$
    \end{itemize}
\end{frame}


\begin{frame}{Bandits in Economics II}


    \begin{itemize}
        \item Caria et al (2021) estimate the effect of cash, information, and psychological 
    incentives on Syrian refugees and local jobseekers in Jordan. 
        \item They use a \textbf{Tempered Thompson Algorithm} to balance maximising treatment 
    precision whilst also maximising experimental participant welfare.
    \end{itemize}

    The algorithm:

    \begin{enumerate}
        \item Generate a prior.
        \item For $t = 1,2,...,n$: sample $\nu_t \sim Q(\cdot | A_1, X_1, ..., A_{t-1}, X_{t-1})$
        \item \begin{itemize}
            \item With probability $\gamma$  sample $A_t$ from $\{1, ..., k\}$
            \item With probability $1 - \gamma$ $A_t = \arg\max_{i \in [k]} \mu_i(\nu_t)$
        \end{itemize} 
    \end{enumerate}

    \begin{exampleblock}{Balancing Precision and Welfare}
        When $\gamma = 1$ we have a traditional RCT, when $\gamma = 0$ we have the 
        Thompson Algorithm. 
        When $0 < \gamma < 1$ TTA asymptotically minimises regret, subject to  
        each treatment has probability of assignment at least $\gamma/k$.
    \end{exampleblock}
\end{frame}

\begin{frame}{Thompson Sampling\footnote{$i$ now indexes individuals, and $k = 1, ..., K$ treatment arms.}}

    Whilst they care about endline outcomes - authors use \textbf{6 week rapid follow up} outcomes as
    $Y_{it}^k$.

    \begin{align*}
       Y_{it}^k | (X_{it} = x, \mu^{kx}, \alpha^k, \beta^k) &\sim Ber(\mu^{kx}) \\
       \mu^{kx} | (\alpha^k, \beta^k) &\sim Beta(\alpha^k, \beta^k) \\
       (\alpha^k, \beta^k) &\sim \pi
    \end{align*}

\end{frame}




\begin{frame}{Inference}

    \begin{itemize}
        \item Fortunately, inference is "easy" in TTA and we can just read off the posterior 
    used for treatment assignment.
        \item The authors also perform RI as an alternative.
    \end{itemize}

\end{frame}


\begin{frame}{Problems In This Setting I}


    \begin{figure}[htbp]
        \centering
    %    \includegraphics[scale=0.2]{plots/ramadan.png} 
        \caption{Non-stationary reward distribution?}
    \end{figure}
\end{frame}

\begin{frame}{Problems In This Setting II}

    \begin{figure}[htbp]
        \centering
    %    \includegraphics[scale=0.15]{plots/tta-assignment.png} 
       \caption{Intermediate outcome chosen wasn't particularly indicative of 
       final outcome.}
    \end{figure}
    

\end{frame}

\begin{frame}{Problems In This Setting III}

   \begin{figure}[htbp]
       \centering
    %   \includegraphics[scale=0.15]{plots/good-tta-assignment.png} 
   \end{figure} 

       Simulations using \textbf{two month} outcomes.
\end{frame}


\begin{frame}{Happy RCT Participants?}

    \begin{figure}[htbp]
        \centering
    %    \includegraphics[scale=0.3]{plots/welfare.png} 
    \end{figure}

    \begin{itemize}
        \item $\Delta_1$ compares welfare from realised assignment to random assignment
        \item $\Delta_2$ compares welfare from optimal targeted (i.e. policy per stratum) policy to no intervention.
        \item $\Delta_3$ compares welfare from optimal non-targeted (i.e. policy fixed across strata) policy to no intervention.

    \end{itemize}
\end{frame}



\section{Adaptive Trials and Preferences}

\begin{frame}{Adaptive trials and Preferences}
        \begin{itemize}
            \item Are we really maximising welfare or just $S = \sum_t^T\sum_i^N Y_{i,t}$?
            \item How do we choose which outcome we care about?
             \begin{itemize}
                 \item Either across distinct measures?
                 \item Or across timeframes like Caria et al?
             \end{itemize}
             \item Are we using the econometrician's loss function or the participant's? 
    \end{itemize}
\end{frame}



\begin{frame}{Towards a Solution}







    

\end{frame}

%------------------------------------------------
\section{Adaptive Trials with}
%------------------------------------------------

\begin{frame}{Table}
    \begin{table}
        \begin{tabular}{l l l}
            \toprule
            \textbf{Treatments} & \textbf{Response 1} & \textbf{Response 2} \\
            \midrule
            Treatment 1         & 0.0003262           & 0.562               \\
            Treatment 2         & 0.0015681           & 0.910               \\
            Treatment 3         & 0.0009271           & 0.296               \\
            \bottomrule
        \end{tabular}
        \caption{Table caption}
    \end{table}
\end{frame}

%------------------------------------------------

\begin{frame}{Theorem}
    \begin{theorem}[Mass--energy equivalence]
        $E = mc^2$
    \end{theorem}
\end{frame}

%------------------------------------------------

\begin{frame}{Figure}
    Uncomment the code on this slide to include your own image from the same directory as the template .TeX file.
    %\begin{figure}
    %\includegraphics[width=0.8\linewidth]{test}
    %\end{figure}
\end{frame}

%------------------------------------------------

\begin{frame}[fragile] % Need to use the fragile option when verbatim is used in the slide
    \frametitle{Citation}
    An example of the \verb|\cite| command to cite within the presentation:\\~

    This statement requires citation \cite{p1}.
\end{frame}

%------------------------------------------------

\begin{frame}{References}
    % Beamer does not support BibTeX so references must be inserted manually as below
    \footnotesize{
        \begin{thebibliography}{99}
            \bibitem[Smith, 2012]{p1} John Smith (2012)
            \newblock Title of the publication
            \newblock \emph{Journal Name} 12(3), 45 -- 678.
        \end{thebibliography}
    }
\end{frame}

%------------------------------------------------

\begin{frame}
    \Huge{\centerline{\textbf{The End}}}
\end{frame}

%----------------------------------------------------------------------------------------

\end{document}