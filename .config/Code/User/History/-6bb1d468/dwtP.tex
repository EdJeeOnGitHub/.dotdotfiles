\documentclass{article}
\author{Edward Jee}

\title{Problem Set 4}

\begin{document}

\maketitle


\subsection*{a}
A surprising puzzle in development, and in economics generally, is the presence 
of publication bias. If economists were rational Bayesian updaters they would 
seek to publish all rigorously derived results and update their posterior beliefs 
about a topic based on the distribution of research results. However, in reality 
there is a bias towards significant results being published in economics journals 
as documented by Brodeur et al. (2016) and Andrews and Kasy (2019). This means 
economists only observe a selected sample of signals and must update their posterior 
beliefs after deconvolving publication bias and sampling variation.
In short, it doesn't make sense for economists to prefer publishing significant results 
since it makes learning about the world harder. 


\subsection*{b}

Standard economic explanations are that significant results are somehow novel 
or new and people derive utility from novelty - both journal editors and 
submitting economists. We could also think of a 
career concerns type model where significant results signal competence and thus 
economists only submit papers with significant results in papers to try and signal 
their type.








    
\end{document}