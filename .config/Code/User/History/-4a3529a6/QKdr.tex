%----------------------------------------------------------------------------------------
%	PACKAGES AND THEMES
%----------------------------------------------------------------------------------------
\documentclass[aspectratio=169,xcolor=dvipsnames]{beamer}
\usetheme{SimplePlus}

\usepackage{hyperref}
\usepackage{graphicx} % Allows including images
\usepackage{booktabs} % Allows the use of \toprule, \midrule and \bottomrule in tables

%----------------------------------------------------------------------------------------
%	TITLE PAGE
%----------------------------------------------------------------------------------------

\title[Experimenter Demand]{Leveraging Experimenter Demand Effects} % The short title appears at the bottom of every slide, the full title is only on the title page

\author[Ed Jee] {Edward Jee}

\institute[UChicago] % Your institution as it will appear on the bottom of every slide, may be shorthand to save space
{
%    Class Presentation 
}
\date{\today} % Date, can be changed to a custom date


%----------------------------------------------------------------------------------------
%	PRESENTATION SLIDES
%----------------------------------------------------------------------------------------

\begin{document}

\begin{frame}
    % Print the title page as the first slide
    \titlepage
\end{frame}

\begin{frame}{Overview}
    % Throughout your presentation, if you choose to use \section{} and \subsection{} commands, these will automatically be printed on this slide as an overview of your presentation
    \tableofcontents
\end{frame}

%------------------------------------------------
\section{Experimenter Demand Effects in the Lab}
%------------------------------------------------

\begin{frame}{Experimenter Demand Effects}
    
    Experimenter demand effects (EDEs)  occur when participants know they're being experimented upon 
    and change their behaviour.

    \vfill
    EDEs induce external validity issues as behaviour would be different in 
    absence of the experimenter.

    Some examples:
    \begin{itemize}
        \item "Authority" figure effect (French and Raven 1959)
        \item Expectations of reciprocity from experimenter
        \item Social desirability bias
        \item Anonymity or lack thereof
    \end{itemize}
\end{frame}
%------------------------------------------------


\begin{frame}{EDE Solutions}


Many experiments, either in the lab or the field, try and shutdown 
    experimenter demand channels:
\begin{itemize}
        \item Obfuscated follow-up - follow-ups presented as independent study to participants.
        \item List methods - veil individual respondent answers.
        \item Neutral framing - experiment instructions shouldn't draw attention to the experiment's purpose.
        \item Obfuscated information - treatment information is revealed alongside irrelevant information.
        \item de Quidt et al. (2018) provide bounds for estimating bias induced by "demand 
    treatments" 
\end{itemize}

\end{frame}



\begin{frame}{How bad are EDEs?}

To measure EDE magnitudes de Quidt et al. create a "strong" and "weak" positive/negative demand effect 
    across 11 lab games on MTurk.

    \begin{itemize}
        \item \textbf{Weak:}  "We expect that participants who are shown these instructions will [work, invest, ...] more/less than they
        normally would."
        \item  \textbf{Strong:}  "You will do us a favor if you [work, invest, ...] more/less than
you normally would."
    \end{itemize}

    

\end{frame}

\begin{frame}[label=magnitudes]{EDE Magnitudes II}

    \begin{figure}[htbp]
        \centering
        \includegraphics[scale=0.45]{quidt-fig.png} 
    \end{figure}
    \hyperlink{res-table}{\beamerskipbutton{Result Table}}
\end{frame}


\begin{frame}{EDE Magnitudes III}

    They label effects as moderate to small:
    
    \begin{itemize}
        \item  "Weak" treatment induced a 0.13 standard deviation change on average.
        \item "Strong" treatment induced an average of 0.6 standard deviations with a 
        range of 0.23 to 1.06 SDs.
    \end{itemize}



    We have less evidence of this effect in the field where stakes are much 
    higher:

    \begin{itemize}
        \item Dictator game involved splitting \$1.
        \item Value of time elicited using 10 cents today or in a week's time
    \end{itemize}
\end{frame}


\begin{frame}{My Goal}

    If all it takes to generate larger treatment effects is the 
    statement "You will do us a favor if you..." from a policy point of view, this is a relatively cheap addition to any 
    intervention.

    \vfill
    Therefore, my goal is to quantify how best to create experimenter demand effects in a field 
    setting that can be scaled to a wider population. 

\end{frame}

\section{Experimenter Demand Effects in the Field}


%------------------------------------------------

\begin{frame}{}
    The lab experiment suggests:
    \begin{itemize}
            \item Women are more likely to try and please an interviewer than men (respond 0.15 SD 
            more than men in the strong treatment).
            \item Stating the experimenter's preferences are more effective than the 
            experiment hypothesis ("You will do us a favor..." vs "We expect that...")
    \end{itemize}
\end{frame}

%------------------------------------------------
\section{Second Section}
%------------------------------------------------

\begin{frame}{Table}
    \begin{table}
        \begin{tabular}{l l l}
            \toprule
            \textbf{Treatments} & \textbf{Response 1} & \textbf{Response 2} \\
            \midrule
            Treatment 1         & 0.0003262           & 0.562               \\
            Treatment 2         & 0.0015681           & 0.910               \\
            Treatment 3         & 0.0009271           & 0.296               \\
            \bottomrule
        \end{tabular}
        \caption{Table caption}
    \end{table}
\end{frame}

%------------------------------------------------

\begin{frame}{Theorem}
    \begin{theorem}[Mass--energy equivalence]
        $E = mc^2$
    \end{theorem}
\end{frame}

%------------------------------------------------

\begin{frame}{Figure}
    Uncomment the code on this slide to include your own image from the same directory as the template .TeX file.
    %\begin{figure}
    %\includegraphics[width=0.8\linewidth]{test}
    %\end{figure}
\end{frame}

%------------------------------------------------

\begin{frame}[fragile] % Need to use the fragile option when verbatim is used in the slide
    \frametitle{Citation}
    An example of the \verb|\cite| command to cite within the presentation:\\~

    This statement requires citation \cite{p1}.
\end{frame}

%------------------------------------------------

\begin{frame}{References}
    % Beamer does not support BibTeX so references must be inserted manually as below
    \footnotesize{
        \begin{thebibliography}{99}
            \bibitem[Smith, 2012]{p1} John Smith (2012)
            \newblock Title of the publication
            \newblock \emph{Journal Name} 12(3), 45 -- 678.
        \end{thebibliography}
    }
\end{frame}

%------------------------------------------------

\begin{frame}
    \Huge{\centerline{\textbf{The End}}}
\end{frame}

%----------------------------------------------------------------------------------------


\begin{frame}[label=res-table]

    \begin{figure}[htbp]
        \centering
        \includegraphics[scale=0.45]{quidt-table.png} 
    \end{figure}
    \hyperlink{magnitudes}{\beamerskipbutton{Return}}

\end{frame}
\end{document}