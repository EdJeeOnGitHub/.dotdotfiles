\documentclass{article}


\setlength\textwidth{135.5mm}
\setlength\textheight{200mm}
\setlength\topmargin{-10pt}
\setlength\oddsidemargin{14mm}
\setlength\evensidemargin{\oddsidemargin}
\setlength\headheight{24pt}
\setlength\headsep   {14pt}
\setlength\topskip   {11.74pt}
\setlength\maxdepth{.5\topskip}
\setlength\footskip{15pt}

\author{Edward Jee}

\title{
    Summary: Human Capital And Cognition 
}
\begin{document}



\maketitle

\section{Contribution}





This paper explores the effect of schooling, or human capital accumulation more 
broadly, on cognitive persistence. The authors hypothesise that increased practice 
at sustained cognitive activity will increase students' ability to focus on 
challenging cognitive tasks - even if these practice activities are completely unrelated 
to the task at hand. First, they document significant evidence of cognitive fatigue,
questions are randomly ordered on PISA tests and later questions are less likely 
to be answered successfully. Furthermore, low income and minority students perform 
even worse on later questions.


To test if increased cognitive practice improves ability on tests, the authors 
create two treatment arms. One treatment arm includes 15 minutes of focused study on math 
problems, the other treatment arm purely uses non-verbal and non-numerical 
puzzles for students to solve over 15 minutes. The control arm has non-structured 
exercises they can choose to solve from the board.


If the effect of treatment operates purely through increased familiarity with 
numerical questions or accumulation of "direct" human capital, we'd expect only 
the math puzzle treatment arm to improve test scores. However, the authors find the 
same effect across the math puzzle and game puzzle intervention - it seems that 
practicing cognition for extended periods of time improves students cognitive 
endurance.

\section*{Importance}


The paper is important as it suggests there are returns to cognitive exercise 
independent of the content being studied - it also suggests sustained cognition is 
a trainable skill.



\section*{Likes/Dislikes}



I like how the paper combines a natural experiment, using date of birth as an 
RD cutoff, as well as an RCT to tease out the effects of increased cognitive 
activity on cognitive endurance.



    
\end{document}


