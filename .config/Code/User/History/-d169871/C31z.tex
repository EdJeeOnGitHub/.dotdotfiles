\documentclass{article}
\setlength\textwidth{135.5mm}
\setlength\textheight{200mm}
\setlength\topmargin{-10pt}
\setlength\oddsidemargin{14mm}
\setlength\evensidemargin{\oddsidemargin}
\setlength\headheight{24pt}
\setlength\headsep   {14pt}
\setlength\topskip   {11.74pt}
\setlength\maxdepth{.5\topskip}
\setlength\footskip{15pt}

\title{Summary: Familiarity Does Not Breed Contempt}
\begin{document}
\maketitle


\section*{Contribution}


Rao uses a range of evidence to investigate the effects of inter-group contact 
on prosociality and discrimination. By exploiting staggered treatment timing in 
a policy forcing elite private schools to accept poor students and various 
lab in the field experiments the author can estimate how attitudes amongst rich 
students change when they're forced to interact with poorer students. 


Rao's first piece of evidence using a difference-in-differences estimator to 
compare prosociality and prejudice amongst cohorts that are fully rich (i.e. are 
treated late) with mixed cohorts. This gives us an average effect on wealthy 
students of sharing a classroom with poorer students.

Rao's second piece of evidence uses individual variation within the classroom - 
in schools which assign study groups according to alphabetical order some rich 
students are forced to work with poorer students. This variation is useful as 
it uncovers the personal effect for a given student, rather than endogenous 
adjustment of classroom politics or curricula.

Overall, the author finds that interacting with poorer students makes richer 
students more prosocial - they're more likely to attend charitable events organised 
by the school for charities helping disadvantaged students.

As an additional test, the paper uses a lab in the field approach and varies 
incentives for students to win a relay contest. When stakes are low, richer 
students are willing to choose slower, but also rich, students for their relay team. 
However, as stakes rise the richer students choose the most able runners, at 
the expense of picking only rich peers. By varying the stakes for winning, Rao 
can estimate WTP for homophily.


\section*{Importance}

The paper is important as it speaks to both sides of the policy debate surrounding 
widening elite private school access to poor children. Whilst the paper shows there 
is some merit to parents' concerns that richer students' learning outcomes suffer 
overall these results aren't strong and only English shows signs of disimprovement.
On the other hand, the paper shows there's some truth to proponents of the policy's 
arguments that richer students benefit from interaction with poorer students - 
they become less prejudiced and more prosocial


\section*{Likes and Dislikes}

I liked how the paper aggregates evidence from a range of sources, using 
natural variation at the school and individual level as well as lab style 
experiments to support its argument.

On the other hand, it would have been cool to take this more seriously - how 
should we aggregate all this evidence into a coherent whole. We have all the 
ingredients but stop short of combining evidence appropriately I feel.











    
\end{document}