\documentclass{article}



\setlength\textwidth{135.5mm}
\setlength\textheight{200mm}
\setlength\topmargin{-10pt}
\setlength\oddsidemargin{14mm}
\setlength\evensidemargin{\oddsidemargin}
\setlength\headheight{24pt}
\setlength\headsep   {14pt}
\setlength\topskip   {11.74pt}
\setlength\maxdepth{.5\topskip}
\setlength\footskip{15pt}






\usepackage{amsfonts}
\usepackage{amsmath}
\usepackage{graphicx}
\title{mCCT Impact Evaluation}
\begin{document}
\maketitle


Our primary goal is to estimate the effect of mobile Conditional Cash Transfers (mCCTs)
on the number of vaccinations\footnote{Penta-1 and Measles-1 have been identified as 
our main focus} administered in 7 districts in Sindh\footnote{TODO: List 
districts here.}. As a secondary goal, we'd like to estimate to what extent these 
estimates are externally valid, and whether we can extrapolate estimated treatment 
effects to similar districts not in the RCT. 


\section{Data}
Due to the ZM data, we have 



\section{Estimation}


\textit{It's not entirely clear to me what level of data aggregation we're going to have 
access to - for now let's index $i$ for individuals, UCs, whatever.}


\subsection*{The Simplest Case: TWFE}
The simplest way to estimate the effect of mCCTs on vaccination numbers is a 
"static" TWFE-regression:


\begin{align*}
   Y_{idt} &= \delta_d + \tau_t + \beta \text{treat}_{idt} + \varepsilon_{idt} 
\end{align*}

Under a parallel trends assumption and homogeneous effects, we interpret $\hat{\beta}$ as the difference-in-differences estimator 
and the programme $ATT$. However, in the presence of staggered treatment timing 
(which we face, as each district is rolled into treatment at different times) the 
static TWFE regression makes "forbidden" comparisons. As described by Goodman-Bacon, rather than comparing treated 
units to un-treated units we compare later treated units to earlier treated units.

Figure \ref{fig:ca} shows a stylised example with staggered treatment timing - 
the red line shows outcomes for the "early" treated group whilst the 
green line shows outcomes for the "late" treated group. Comparing 
$\Delta_a$ to $\Delta_b$ is a "good" comparison as it uses untreated 
units as a control for treated units. Using $\Delta_b$ and $\Delta_c$ to make 
comparisons is "forbidden", we're comparing units treated earlier to 
those treated later.

\begin{figure}[htbp]
   \centering
  \includegraphics[scale=0.2]{plots/ca-example.png} 
   \caption{An event study with staggered treatment timing.}
   \label{fig:ca}
\end{figure}

There's been a methodological explosion dealing with these issues and next we 
turn to a proposed solution.

\subsection*{Slightly Less Simple: Callaway and Sant'Anna}
\subsubsection*{ATTs}
Callaway and Sant'Anna suggest a very simple solution. At its heart, difference-in-differences
is comparing two differences in means. CA's proposal is just to calculate these means
whilst keeping careful track of the comparisons we're making. They create the 
variable $g \in \{ \text{Jan, Feb, ..., Dec}\}$ which corresponds to when a group 
is first treated. Then, they propose the target parameter:


\begin{align*}
   ATT(g,t) &= E\left[ 
      Y_t(g) - Y_t(0) | G_g = 1
   \right]
\end{align*}

This is the difference between being treated, $Y_t(g)$, and 
not being treated, $Y_t(0)$, for each group $g$ at each time period $t$. Unfortunately, 
we only ever observe one of $Y_t(g), Y_t(0)$ - this is where the parallel trends 
assumption comes in:
\begin{align*}
   E\left[Y_t(0) - Y_{t-1}(0) | \underbrace{G_g = 1}_{\text{{\tiny Units first treated in $g$.}}}\right] &= 
   E\left[
      Y_t(0) - Y_{t-1}(0) | \underbrace{G_g  = 0 , D_s = 0}_\text{{\tiny Units not yet treated.}} 
   \right],  \\ \forall g = 2, ... \mathcal{T}, & (s,t) = 2, ..., \mathcal{T} \text{ s.t. } t \geq g
\end{align*}

    i.e. in absence of initial treatment in period $g$, units would 
    have the same outcome path as those not yet treated. 

   Without covariates this leads to a very simple comparison estimator:

   \begin{align*}
      \hat{ATT}(g,t) = \frac{1}{N_g} \sum_{i=1}^{N_g}  (Y_{t,i} - Y_{g-1,i}) \mathbb{I}\{G_{gi} = 1\} -
      \frac{1}{N_{nyt(g)}} \sum_{i=1}^{N_{nyt(g)}} (Y_{t,i} - Y_{g-1,i})\mathbb{I}\{nyt(g)_i\} 
   \end{align*}


   We can do this easily with the 7 districts in the RCT and estimate $ATT$s for 
   6 groups \footnote{We have to drop the last district treated as we have no valid comparison 
   group for it.} and $\approx$ 24 time periods.
   
  \subsubsection*{Treatment Effects} 
   Now we have our $ATT$ building blocks we can construct actual parameters of
   interest. These essentially take some weighted sum of the $ATT$s in a way that 
   makes "economic" sense. They all take the form of:

   \begin{align*}
      \theta &= \sum_{g \in \mathcal{G}}\sum_{t=2}^\mathcal{T} w(g,t)\cdot ATT(g,t)
   \end{align*}

   We're probably interested in $\theta_{es}(e)$ - the "event study" estimand - that 
   describes how average treatment effects vary with length of \textit{exposure} or 
   \textit{event time}.

   For instance, Figure  \ref{fig:att} shows each group's $ATT$ using calendar time, 
   the event study estimand, $\theta_{es}$ displayed in Figure \ref{fig:es}, aggregates each $ATT$ relative to 
   \textit{event time} - the first period a district was treated. 


   \begin{figure}[htbp]
      \centering
     \includegraphics[scale=0.4]{plots/att-example.png} 
      \caption{Simulated $ATT(g,t)$s}
      \label{fig:att}
   \end{figure}

   \begin{figure}[htbp]
      \centering
     \includegraphics[scale=0.4]{plots/es-example.png} 
      \caption{Simulated $ATT(g,t)$s}
      \label{fig:es}
   \end{figure}

   Another parameter we might care about is averaging across time and asking 
   what is a single parameter to describe each district's treatment effect:
   \begin{align*}
      \theta_{sel}(\tilde{g}) &= \frac{1}{\mathcal{T} - \tilde{g} + 1} \sum_{t = \tilde{g}}^\mathcal{T} ATT(\tilde{g}, t)
   \end{align*}

   However, this still leaves us with $G-1$ parameters. If we want a single 
   summary parameter of the effect of the programme, we might like to consider:

   \begin{align*}
      \theta_{sel}^O = \sum_{g \in \mathcal{G}} \theta_{sel}(g) P(G = g | G \leq \mathcal{T})
   \end{align*}
   which averages over each district's summary treatment effect, weighted by the 
   probability a district is treated at a certain point (for us this is just $1/7$ 
   due to random assignment).


   An alternative would be to simply sum over all the $ATT(g,t)$s and longer 
   treated districts have more of an impact on the overall estimand, rather than simple 
   averaging across districts:

   \begin{align*}
      \theta_W^O &= \frac{1}{\kappa} \sum_{g \in \mathcal{G}} \sum_{t=2}^\mathcal{T} 
      \mathbb{I}\{t \geq g\} ATT(g,t) P(G = g | G \leq \mathcal{T})
   \end{align*}



   \subsection*{Using Non-Experimental Data}

   Since we have data on every district in Sindh we can think about what assumptions 
   we'd need to use these districts in our difference-in-differences estimator.

   \subsubsection*{A big parallel trends assumption}
   We know that the seven selected districts were chosen because they have the 
   lowest vaccination rates using Measles-1 and Penta-3 data from August (TODO year here). This 
   creates problems for the external validity of estimates, if we believe 
   the level of vaccination will be correlated with the size of treatment effects 
   we're only observing effects for a selected sample. However, this doesn't create 
   issues for the \textit{internal} validity of our estimates since within these 
   seven districts treatment timing is still completely random.

   Fortunately, under some assumptions, this also shouldn't violate  our 
   parallel trends assumption. This 
   is because selection is on \textit{levels} and not \textit{differences}.  
   Provided levels aren't correlated with differences then selection on levels 
   doesn't matter and we can use the remaining districts in our not-yet-treated 
   group.

   \textbf{Thoughts:}

   This is a pretty heroic exercise: we need to assume parallel trends holds across 
   all 30 districts. There's also some tension in the belief that vaccination 
   levels will be correlated with treatment effects whilst also not being correlated 
   with differences.


   The good news is that we have lots of data to test pre-trends and we can even 
   test whether levels are correlated with differences or treatment effects.\footnote{Although 
   Roth (2020) shows that such pre-trend tests aren't always ideal.}

   \subsubsection*{A smaller assumption}

   Using direct knowledge of the site selection mechanism we can estimate the 
   probability a district was included in the RCT sample of seven districts. Since the 
   decision to use Measles-1 and Penta-3 as the vaccination metrics was taken 
   orthogonal to beliefs about treatment effects or parallel trends we can 
   treat this as a "as good as random" choice.  

   By permuting through alternative choices of vaccination metrics and observing 
   how often a district is included in the RCT due to having the lowest vaccination 
   rate, for that set of metrics, we can directly calculate the propensity 
   score for RCT inclusion.
   
   In practice when we do this only five additional districts satisfy the overlap 
   condition, $0 < p(X) < 1$, and have inclusion probabilities not equal to 0 or 1. 

   Using this propensity score now requires a conditional parallel trends assumption:

   


\begin{align*}
   E\left[Y_t(0) - Y_{t-1}(0) |\  p(X), G_g = 1\right] &= 
   E\left[
      Y_t(0) - Y_{t-1}(0) |\  p(X), G_g  = 0 , D_s = 0
   \right],  \\ \forall g = 2, ... \mathcal{T}, & (s,t) = 2, ..., \mathcal{T} \text{ s.t. } t \geq g
\end{align*}


Our new assumption is: Conditional on the probability of selection into the RCT, 
all 12 districts would have had the same path of untreated outcomes. Whilst 
we have institutional knowledge that sites weren't selected in relation to 
parallel trends we can control directly for selection and alleviate some concerns. 

The new estimator is a little cumbersome:

\begin{align*}
  ATT(g,t) &= E\left[
     \left(
   \frac{G_g}{E[G_g]} -
   \frac{\frac{
      p_{g,t}(X)(1 - D_t)(1 - G_g)
   }{
      1 - p_{g,t}(X)
   }}{
  E\left[    \frac{
   p_{g,t}(X)(1 - D_t)(1 - G_g)
      }{1 - p_{g,t}}
   \right]
      }
     \right)
     (Y_t - Y_{g-t})
  \right] 
\end{align*}


Our permutation procedure directly estimates the propensity score non-parametrically -
we can just plug in out estimate of $p_{g,t}(X)$ for each district.

\textbf{Thoughts:}


We now have a parallel trends assumption that's much easier to stomach - 
five "similar" districts rather than all of Sindh and we can control  
for the selection mechanism.


\section{Extrapolation}



\end{document}