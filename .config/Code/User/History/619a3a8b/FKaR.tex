\documentclass{article}

\usepackage{amsmath}
\title{mCCT Impact Evaluation}
\begin{document}
\maketitle


Our primary goal is to estimate the effect of mobile Conditional Cash Transfers (mCCTs)
on the number of vaccinations\footnote{Penta-1 and Measles-1 have been identified as 
our main focus} administered in 7 districts in Sindh\footnote{TODO: List 
districts here.}. As a secondary goal, we'd like to estimate to what extent these 
estimates are externally valid, and whether we can extrapolate estimated treatment 
effects to similar districts not in the RCT. 


\section{Data}
Due to the ZM data, we have 



\section{Estimation}


\textit{It's not entirely clear to me what level of data aggregation we're going to have 
access to - for now let's index $i$ for individuals, UCs, whatever.}


\subsection*{The Simplest Case: TWFE}
The simplest way to estimate the effect of mCCTs on vaccination numbers is a 
"static" TWFE-regression:


\begin{align*}
   Y_{idt} &= \delta_d + \tau_t + \beta \text{treat}_{idt} + \varepsilon_{idt} 
\end{align*}

Under a parallel trends assumption and homogeneous effects, we interpret $\hat{\beta}$ as the difference-in-differences estimator 
and the programme $ATT$. However, in the presence of staggered treatment timing 
(which we face, as each district is rolled into treatment at different times) the 
static TWFE regression makes "forbidden" comparisons. As described by Goodman-Bacon, rather than comparing treated 
units to un-treated units we compare later treated units to earlier treated units.

There's been a methodological explosion dealing with these issues and next we 
turn to a proposed solution.

\subsection*{Slightly Less Simple: Callaway and Sant'Anna}

Callaway and Sant'Anna suggest a very simple solution. At its heart, difference-in-differences
is comparing two differences in means. CA's proposal is just to calculate these means
whilst keeping careful track of the comparisons we're making.






\end{document}