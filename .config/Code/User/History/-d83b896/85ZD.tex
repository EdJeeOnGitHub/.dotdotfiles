\documentclass{article}



\begin{document}
   

\section{Contract Design}



\subsection*{a}


When $t=1$ arrives the consumer knows she must pay $(c+p)$ this period and will 
earn $b$ next period. Since she's contemporaneously discounting next period's 
consumption she correctly discounts using $\beta \delta$. This gives the 
inequality:

\begin{align*}
    \beta \delta b -c - p > 0
\end{align*}



\subsection*{b}
At $t=0$ the consumer uses her best guess of $\beta$, $\hat{\beta}$, and therefore
her expected decision becomes:

\begin{align*}
   \hat{\beta} \delta b - c - p > 0 
\end{align*}


The difference between $\beta -\hat{\beta}$ is what drives the wedge between her 
expected and actual consumption choice.


\end{document}