\documentclass{article}



\begin{document}
   

\section{Contract Design}



\subsection*{a}
\subsubsection*{1}

When $t=1$ arrives the consumer knows she must pay $(c+p)$ this period and will 
earn $b$ next period. Since she's contemporaneously discounting next period's 
consumption she correctly discounts using $\beta \delta$. This gives the 
inequality:

\begin{align*}
    \beta \delta b -c - p > 0
\end{align*}



\subsubsection*{2}
At $t=0$ the consumer uses her best guess of $\beta$, $\hat{\beta}$, and therefore
her expected decision becomes:

\begin{align*}
   \hat{\beta} \delta b - c - p > 0 
\end{align*}


The difference between $\beta -\hat{\beta}$ is what drives the wedge between her 
expected and actual consumption choice.



\subsubsection*{3}


At $t=0$ she knows she wants to consume if:

\begin{align*}
    \delta b - c - p > 0
\end{align*}
This is because $\beta$ cancels when she considers future intertemporal tradeoffs 
however, once she reaches this time period she discounts the next period by $\beta$ -
her preferences are time inconsistent.

\end{document}