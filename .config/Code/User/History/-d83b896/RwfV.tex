\documentclass{article}

\usepackage{amsmath}
\usepackage{amsfonts}


\begin{document}
   

\section{Contract Design}



\subsection*{a}
\subsubsection*{1}

When $t=1$ arrives the consumer knows she must pay $(c+p)$ this period and will 
earn $b$ next period. Since she's contemporaneously discounting next period's 
consumption she correctly discounts using $\beta \delta$. This gives the 
inequality:

\begin{align*}
    \beta \delta b -c - p > 0
\end{align*}



\subsubsection*{2}
At $t=0$ the consumer uses her best guess of $\beta$, $\hat{\beta}$, and therefore
her expected decision becomes:

\begin{align*}
   \hat{\beta} \delta b - c - p > 0 
\end{align*}


The difference between $\beta -\hat{\beta}$ is what drives the wedge between her 
expected and actual consumption choice.



\subsubsection*{3}


At $t=0$ she knows she wants to consume if:

\begin{align*}
    \delta b - c - p > 0
\end{align*}
This is because $\beta$ cancels when she considers future intertemporal tradeoffs 
however, once she reaches this time period she discounts the next period by $\beta$ -
her preferences are time inconsistent.



$\beta$ is generating present bias since she discounts the next period, $t$ vs 
$t+1$ by $\beta$ whereas for $t', t'+1$ in the future there's no sense of 
presency bias.


Her naivete is leading to the difference between (1) and (2) - if $\hat{\beta} = \beta$
then her expected and actual consumption are identical.


\subsection*{b}


If the consumer doesn't buy the contract she gets discounted utility $\beta \delta \overline{u}$.

If she does buy the contract she receives:

\begin{align*}
    \mathbb{E}[U | \textit{Accept}] &= \mathbb{E}\left[ 
        -L\beta \delta + \beta \delta \max\{\delta b - c - p, 0\} 
    \right] \\
    &= -L \beta \delta + \beta \delta \int^{\hat{\beta}\delta b - p}_{-\infty} \delta b - c - p dF(c)
\end{align*}


There's a difference between her choice equation and realised benefit equation as 
highlighted by question a). At $t=0$ she forecasts she will face intertemporal 
discount rate $\hat{\beta}$ so expects her decision to depend on a realisation of 
$c$ less than $\hat{\beta}\delta b - p$ but in reality if she does take the 
contract and consume she will earn discounted gains of $\beta \delta (\delta b - c - p)$. 



Therefore, the only inquality from $a$ that doesn't enter this decision is the 
first one: $\beta \delta b - p - c$.


Rewriting to isolate $L$:

\begin{align*}
    - \overline{u} + \int^{\hat{\beta}\delta b - p}_{-\infty} \delta b - c - p \\ dF(c) > L
\end{align*}

\subsection*{c}


The monopolist chooses:

\begin{align*}
    \max_{L,p} \delta \left[ L \pi(\textit{Accept}(L, p)) + (p - a) \pi(\textit{accept}(L, p) \cap \textit{consume}(L, p)) \right]
\end{align*}


However, if no consumer accepts the monopolist earns 0 profit so we rewrite the 
problem as maximising profit, conditional on acceptance which requires 
specifying the IR constraint for the agent is satisfied from (b).


Therefore:

\begin{align*}
    \max_{L,p} &\delta \left[
    L + \delta (p-a) F(\beta \delta b - p)
    \right]  \\
    & s.t. \ \left[ 
        -L + \int^{\hat{\beta}\delta b - p}_{-\infty} \delta b - p - c dF(c)
        \right] = \overline{u}
\end{align*}


\end{document}