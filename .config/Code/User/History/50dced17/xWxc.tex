\documentclass{article}

\usepackage[margin=0.5in]{geometry}
\usepackage{bm}
\usepackage{amsmath}
\usepackage{graphicx}
\usepackage{geometry}
 \geometry{
 a4paper,
 total={170mm,257mm},
 left=20mm,
 top=20mm,
 }
 \usepackage{fancyhdr}

 \pagestyle{fancy}
 \fancyhf{}
 \rhead{Overleaf}
 \lhead{Guides and tutorials}
% \title{Adaptive Trials and Preferences}
\author{Edward Jee}
\begin{document}
% \maketitle




Traditionally, adaptive trials focus on a single outcome of concern and attempt to maximise 
$S_t = \sum^t_{0}Y_t$ by varying treatment arm proportions as new data arrives - for example, Caria et al (2021), maximise the 
six-week follow-up employment rate of job seekers in Jordan. However, in 
practice economists and policymakers often observe data on multiple outcomes and must 
choose between distinct outcomes or a timeframe with which to measure outcomes.
    
Taking this decision seriously 
implies the utility function of participants only depends on one outcome or good 
and this particular good is known to the economist. Furthermore, this makes it 
easy to mistake the loss or welfare function of the economist for the trial 
participants - ideally we'd like to elicit the preferences of in-sample participants 
and use this welfare function to inform treatment assignment. 


My goal is  to elicit preferences, characterised by the 
random utility model (RUM), whilst adaptively assigning subjects 
to treatment arms. By jointly modelling uncertainty over marginal rates of substitution 
across goods and treatment effect uncertainty we choose the each arm in proportion 
to which we believe it's optimal.

\subsection*{The "Traditional" Bandit}


Traditional Thompson sampling involves:


    \begin{itemize}
        \item Generate a prior over each arm's rewards.
        \item Take a random draw from this prior distribution and choose the best action given your draw.
        \item Observe the reward associated with your action.
        \item Update your posterior over arm rewards.
        \item Today's posterior is tomorrow's prior $\rightarrow$ return to step two.
    \end{itemize}


However, with multiple goods we observe a  vector  $Y_t = \begin{bmatrix} y^1_t & y^2_t & ... & y^j_t\end{bmatrix}$.
Now, each treatment arm $k$ has expected reward vector $\bm{\mu}_k = \begin{bmatrix}
    \mu^1_k & \mu^2_k & ... & \mu^j_k
\end{bmatrix}'$. We now need to find some function $f(Y_t)$ to maximise. 

\subsection*{The "Utility" Bandit}

Let $U_k = V_k + \varepsilon_k$ be the utility a 
participant receives from \textit{choosing} treatment arm $k$. Assume $V_k = 
\gamma_1 \mu^1_k + \gamma_2 \mu^2_k + ... + \gamma_J \mu^J_k = \bm{\mu}_k\bm{\gamma}'$. Our ideal experiment would be to vary the perceived effectiveness of each treatment arm $k$ for 
outcome $j$ and let variation in treatment effectiveness map out individuals' 
indifference curves over outcomes. Fortunately, the Thompson sampling posterior 
provides us with exactly such variation.


In the language of IO, each treatment arm is a good and its expected effect on outcomes 
is the good's characteristic. In the "welfare sample" we estimate a discrete choice model 
and in the "treatment arm" we estimate a conventional regression model. The estimated treatment effects in the treatment sample 
become the $X$s in the welfare estimation sample.

Therefore each period we:

    \begin{itemize}
        \item Draw $\bm{\mu}_{k, i}$ from the posterior for each individual in the "welfare" arm. 
        \item Report a $\bm{\mu}_{k, i}$ to each individual and observe their choices.
        \item Regress their choices on $\bm{\mu}_{k,i}$ and update posterior over $\gamma$.
        \item Sample from joint posterior and draw $\bm{\mu}_{k}, \gamma$
        \item Choose treatment proportions for the treatment sample according to probability  $\bm{\mu}_k \gamma'$ is 
        greater than $\bm{\mu}_l \gamma', l \neq k$.
        \item Observe rewards across arms and update $\bm{\mu}_k$ posterior.
        \item Return to step 1.
         \end{itemize}


We weight each arm's benefits using the estimated $\bm{\gamma}$ and 
assign participants in the treatment sample to each treatment arm in proportion to its probability of being optimal \textit{
    accounting for posterior beliefs about marginal rates of substitution and 
    posterior beliefs about each treatment arm's expected reward
}. Figure \ref{fig:mab} shows the utility bandit in action.

\begin{figure}[htbp]
    \centering
    \includegraphics[scale=.65]{mab-mrs-ot.png} 
    \label{fig:mab}
    \caption{An "Easy" Example.
     Arm Proportions shown in upper panel. Estimated MRS in lower panel.}
    
\end{figure}

\begin{figure}[htbp]
    \centering
    \includegraphics[scale=.65]{hard-mab-example.png} 
    \label{fig:mab}
    \caption{A "Tricky" Example.
     Arm Proportions shown in upper panel. Estimated MRS in lower panel.}
\end{figure}



\end{document}