\documentclass{article}

\usepackage[margin=0.5in]{geometry}
\setlength\textwidth{135.5mm}
\setlength\textheight{200mm}
\setlength\topmargin{-10pt}
\setlength\oddsidemargin{14mm}
\setlength\evensidemargin{\oddsidemargin}
\setlength\headheight{24pt}
\setlength\headsep   {14pt}
\setlength\topskip   {11.74pt}
\setlength\maxdepth{.5\topskip}
\setlength\footskip{15pt}


\title{Adaptive Trials and Preferences}
\begin{document}


Many randomised control trials in economics seek to estimate some parameter of 
interest with minimal disruption to in-sample participants.  An emerging trend in programme 
evaluation, popularised by A/B testing amongst technology firms, is the use of 
"multi-armed bandits" (MABs) or adaptive trials which speak directly to the goal 
of maximising sample welfare. As sample participants are rolled into programmes 
across multiple waves, adaptive trials seek to overweight promising treatment 
arms whilst underweighting arms considered to be less promising.



Traditionally, MABs focus on a single outcome of concern and attempt to maximise 
$S_t = \sum^t_{0}Y_t$ - for example, \cite{adaptive-jordan} maximise the 
two-month follow-up employment rate of job seekers in Jordan. However, in 
practice economists and policymakers often observe data on multiple outcomes and 
must design an optimal policy accounting for individual welfare aggregated across 
a range of outcomes or goods. Traditionally, economists specify a primary outcome 
of interest and use this as a metric for optimal policy decisions. 

    
Taking this decision seriously 
implies the utility function of participants only depends on one outcome or good 
and this particular good is known to the economist. Furthermore, this makes it 
easy to mistake the loss or welfare function of the economist for the trial 
participants - ideally we'd like to elicit the preferences of in-sample participants 
and use this welfare function to inform treatment assignment. 


My goal is  to jointly elicit preferences, characterised by the 
random utility model (RUM) \citep{mcfadden}, whilst adaptively assigning subjects 
to treatment arms. 


\end{document}