\documentclass{article}

\author{Edward Jee}

\title{Research and Interest Statement}

\begin{document}
\maketitle

My research interests are at the intersection of programme evaluation and machine 
learning, particularly with an application towards development economics. One 
of my projects seeks to synthesise observational and experimental evidence to 
estimate the effect of mobile conditional cash transfers (mCCTs) on 
child vaccination rates in Sindh, Pakistan. Whilst we have experimental variation 
in seven districts through an RCT with a staggered roll-out design, we observe 
individual level outcomes for 
the universe of children in the remaining 23 districts. Furthermore, the implementing 
partner chose the seven districts with the lowest vaccination rates as candidates 
for the experimental treatment. Therefore, the goal is to combine causal inference, 
through careful modelling of the site selection mechanism using Borusyak and Hull (2021), 
with machine learning style estimators to improve precision of the RCT estimates 
leveraging "big data" from the remaining observational districts.



Another project aims to take preferences and welfare seriously in multi-arm 
bandits/adaptive trials. Often in development economics we measure a range of 
outcomes from some treatment but only target one outcome to maximise using a 
bandit algorithm. I propose a modified Thompson sampling routine that jointly 
learns participant preferences over outcomes and treatment effects to inform optimal treatment 
allocation. Rather than imposing the econometrician's chosen loss function, as 
the current status quo does, we learn the loss function of participants and 
aggregate across outcomes using estimated marginal rates of substitution. Furthermore, 
since posterior inference is joint, we balance the exploitation-exploration 
tradeoff across uncertainty about outcomes and preferences.


More broadly, I am interested in developing methods to generalise inference from 
randomised control trials, in combination with observational data, to better inform 
policy. Whilst my experience so far has predominantly been from a Bayesian perspective
I'd value the opportunity to learn and apply more frequentist machine learning methods 
to causal inference problems at the Summer Institute.



    
\end{document}